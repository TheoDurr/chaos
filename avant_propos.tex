\chapter*{Avant-propos}
% Explication du choix du sujet, méthodologie de travail
On entend souvent parler du chaos comme l'absence d'ordre ou de logique, comme une force mystérieuse qui 
rendrait aléatoire et imprédictible certains évènements et façonnerait le monde sans que l'on puisse vraiment 
comprendre comment et pourquoi. Le chaos est également souvent associé avec l'effet papillon 
qui explique qu'un changement inperceptible peut avoir des conséquences de proportion complètement différentes.

Ce sont sans doute l'image populaire, le mystère et l'aspect universel du chaos qui nous ont poussé à nous intéresser
 à ce sujet, et à vouloir acquérir de plus amples connaissances à son propos, de plus il s'agit d'un sujet très interessant
  car l'étude des phénomène chaotique est un champ de recherche relativement récent qui touche à de très nombreux domaines et 
  permet d'appréhender le monde d'une manière nouvelle et différente.  Pour se faire, nous avons utilisés diverses méthodes, 
  évidement nous avons mené de nombreuses recherches sur le sujet afin de ce familiariser avec lui et d'avoir une meilleure idée
   de son ampleur et des différents domaines où l'ont peut retrouver l'intervention du chaos. 

Evidement, l'étude de la théorie du chaos dans son ensemble nous est inaccessible de part sa complexité et ses applications très nombreuses.
 Nous avons donc fait le choix d'étudier des phénomènes mathématiques et physiques relativement simples ou l'on peut retrouver la marque du chaos. 
 Ses études passent bien sûr par une analyse mathématique mais également par la construction de graphiques à l'aide d'algorithmes que nous avons écrit
  afin de mettre en application notre compréhension et nos connaissances de ces phénomènes.

