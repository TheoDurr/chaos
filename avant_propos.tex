\chapter*{Avant-propos}
% Explication du choix du sujet, méthodologie de travail
On entend souvent parler du chaos comme l'absence d'ordre ou de logique, comme une force mystérieuse qui 
rendrait aléatoire et imprédictible certain évènement et façonnerait le monde sans que l'on puisse vraiment comprendre comment et pourquoi. Le chaos est également souvent associé avec l'effet papillon 
qui explique qu'un changement inperceptible peut avoir des conséquences de proportion completement différentes.
Ce sont sans doute l'image populaire, le mystère et l'aspect universel du chaos qui nous ont poussé à nous intéresser à ce sujet, 
et a vouloir acquérir de plus ample connaissance à son propos, de plus il s'agit d'un sujet très interessant car l'étude des phénomène 
chaotique est un champ de recherche relativement récent qui touche à de très nombreux domaine et permet d'appréhendé le monde d'une manière nouvelle et différente.
 Pour ce faire, nous avons utilisés diverses méthodes, évidement 
nous avons mené de nombreuse recherche sur le sujet afin de ce familiarisé avec lui et d'avoir une meilleure idée de son ampleur
et des différents domaines ou l'ont peut retrouver l'intervention du chaos. Evidement l'étude de la théorie du chaos dans son ensemble
nous est inaccessible de part sa complexité et ses applications très nombreuses. Nous avons donc fait le choix d'étudier des phénomènes
mathématiques et physiques relativement simple ou l'on peut retrouver la marque du chaos. Ses études passe bien sûr par une analyse mathématiques
mais également par la construction de graphique à l'aide d'algorithme que nous avons écrit afin de mettre en applications notre compréhension 
et nos connaissance de ses phénomènes.

