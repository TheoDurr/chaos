\newcommand{\HRule}{\rule{\linewidth}{0.1mm}} % Ligne horizontale (épaisseur modifiable}
\enlargethispage{2cm} % Réduit la taille du footer

\begin{center}
    \includegraphics[width=7cm]{utbm_logo.jpg} \\

    % Titre
    \HRule \\[0.6cm]
    \begin{flushleft}
        \huge\textbf{Mots clefs}
    \end{flushleft}

    Fractales - Diagramme de bifurcation - Lorenz - Verhlust - Chaos - Effet Papillon - Malkus - Fonction logistique - Attracteur - Point fixe stable et instable - Python - Suite logistique

    \vspace{1cm} 

    \begin{flushleft}
        \huge\textbf{Résumé}
    \end{flushleft}

    Ce rapport inities aux bases de l'études des systèmes dynamiques non déterministes, par l'étude de la fonction logistique, utilise dans la modélisation de populations; ou encore les attracteurs de lorenz, utilisés dans l'étude de l'atmosphère terrestres.

    \vspace{1cm} 
    \HRule \\[1.5cm]

    \vfill

    \large Théo DURR \& Célian HUMBERT

\end{center}