\chapter{Introduction}
A travers de nos recherches nous avons essayé de comprendre et d'étudier
le chaos en passant par l'étude de systèmes dynamiques non déterministes, bien que cet énoncé 
puisse paraitre compliqué, il désigne simplement tous systèmes dynamique dont on ne peut pas déterminer 
précisément l'évolution sans une connaissance parfaite de ses conditions initiales, il s'agit de la
principale caractéristiques des systèmes chaotique. Nous avons notamment étudié deux célèbres systèmes
exhibant ces caractéristiques : La fonction logistique et les célèbres équations de Lorenz qui sont l'un des
point de départ de la théorie du chaos et à qui sont également à l'origine de l'expression "effet papillon" 
car la conférence concernant ses travaux avait pour titre "Does the flap of a butterfly’s wings in Brazil set off a tornado in Texas ?"
Nous essayerons également d’apporter des éléments de réponse à cette célèbre question.
Les deux systèmes mentionnés précédemment présente une forte dépendance aux conditions initiales, mais il possède une autre 
caractéristique que l'on retrouve souvent dans les systèmes chaotiques: ils possèdent une dimension fractale.
Vous trouverez donc tout au long de ce document l'analyse et l'explication du comportement de ses systèmes afin de mieux comprendre
ce qu'est "scientifiquement" le chaos.
  
