\chapter{Introduction}
Au travers de nos recherches, nous avons essayé de comprendre et d'étudier le chaos, ou appelés mathématiquement les systèmes dynamiques non déterministes. 

Bien que cet énoncé puisse paraître compliqué, il désigne simplement tout système dynamique dont on ne peut pas déterminer précisément l'évolution sans une connaissance parfaite de ses conditions initiales, il s'agit de la principale caractéristique des systèmes chaotiques. Nous avons notamment étudié deux célèbres systèmes exhibant ces caractéristiques : La fonction logistique, ainsi que les célèbres équations de Lorenz qui sont l'un des \og point de départ \fg{} de la théorie du chaos et à qui sont également à l'origine de l'expression "effet papillon".

L'origine de ce nom vient d'une de ses conférences concernant ses travaux, et qui avait pour titre "Does the flap of a butterfly’s wings in Brazil set off a tornado in Texas ?" (Traduit par \og Le battement d'ailes d'un papillon au Brésil peut-il déclencher une tornade au Texas ?\fg{}). Nous essayerons d’apporter des éléments de réponse à cette célèbre question. Les deux systèmes mentionnés précédemment présentent une forte dépendance aux conditions initiales, mais ils possèdent une autre caractéristique que l'on retrouve souvent dans les systèmes chaotiques: ils possèdent une dimension fractale. Vous trouverez donc tout au long de ce document l'analyse et l'explication du comportement de ses systèmes afin de mieux comprendre ce qu'est "scientifiquement" le chaos.
  
