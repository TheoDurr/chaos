\chapter{Les attracteurs de Lorenz}
Tout d’abord commençons par définir rapidement et sans rentré dans les détails mathématiques ce qu’est un attracteur,
 il s’agit d’un ensemble ou d’un espace vers lequel un système dynamique évolue de manière
 irréversible en l’absence de perturbation, le concept d’attracteur est l’une des bases de la théorie du chaos. 
 Dans cette partie nous allons nous concentrer sur l’attracteur de Lorenz, il s’agit d’un attracteur
représentant l’évolution du système dynamique différentiel de Lorenz. Cet attracteur est caractérisé d'attracteur étrange, cela signifie qu'il n'est pas continue, mais formé point par point par la dynamique du système.\\

Le système de Lorenz s'écrit: 
\[
    \left\{
    \begin{array}{rcl}
        \dfrac{dx}{dt}&=&\sigma[y-x]\\
        \dfrac{dy}{dt}&=&x(\rho-z)-y\\
        \dfrac{dz}{dt}&=&xy-\beta z\\
    \end{array}
    \right.
\]
Ces équations sont un modèle très simplifier crée par Lorenz pour modélisé le fonctionnement 
de l’atmosphère terrestre. Il a décidé de cherché un modèle simple car 
les équations décrivant l’atmosphère de façon précise était beaucoup trop compliqué à résoudre
 numériquement à l’époque de Lorenz et il souhaitais pouvoir étudié  le phénomène de convection
de Rayleigh-Bénard à l’aide de son modèle simplifié.\\


Dans ces équations: \\
$\sigma$  correspond au nombre de Prandtl, il s'agit d'un nombre sans dimension obtenu en calculant le rapport entre la diffusivité de la quantité de mouvement et ça diffusivité thermique.\\\\
$\rho$    correspond au nombre de Rayleigh, il s'agit d'une valeur utilisé en mécanique des fluides, il permet de caractérisé le transfert de chaleur au sein d'un fluide \\\\
$\beta $  est une valeur dépendand de la couche de l'atmosphère\\\\
$x$, $y$, $z$ représente l'état du système au court du temps, remis dans leur context physique, $x$ est proportionnel à l'intensité du mouvement de convection, $y$ est proportionnel à la différence de température entre les courants ascendants et descendants. $z$ est proportionnel à l'écart du profil de température vertical par rapport à un profil linéaire.    

\section{Dimension mathématique}

\subsection{Equilibres du modèle}
Nous allons nous intéresser aux point auxquelles le modèle est stable, c’est-à-dire les points d’équilibre $(x,y,z)$ vérifiant $x'=y'=z'=0$, cela revient a résoudre le syteme suivant :\\
\[
    \left\{
    \begin{array}{rcl}
        \sigma(y-x)&=&0\\
        x(\rho-z)-y&=&0\\
        xy-\beta z&=&0\\
    \end{array}
    \right.
\]\\

Avec $\sigma$, $\rho$ et $\beta$ des réels positif.

On remarque que $x=y=z=0$ est une solution évidente de ce système quelque soit les valeurs de $\sigma$, $\rho$ et $\beta$.
A l'aide d'un logiciel de calcule formel, on résoud ce système et l'on trouve des solutions dépendand de $\sigma$, $\rho$, $\beta$ et particulièrement de la valeur de $\rho$, en effet on trouve:\\

Si $\beta(\rho-1)\geq 0$ alors $x= \pm \sqrt{(\beta(\rho-1)}$, $y= \pm \sqrt{(\beta(\rho-1)}$ et $z=\rho-1$\\
autrement dit lorsque $\rho\geq 1$ le système admet 3 points d'équilibre à savoir:\\
 $(\sqrt{(\beta(\rho-1)}; \sqrt{(\beta(\rho-1)}; \rho-1)$, $(-\sqrt{(\beta(\rho-1)}; -\sqrt{(\beta(\rho-1)}; \rho-1)$ et $(0,0,0)$\\\\
 dans le cas ou $\rho < 1$ il admet un seul point d'équilibre à savoir $(0,0,0)$\\
 
 \subsection{Stabilité des points d'équilibre}
Les point d'équilibre trouvé précédment ne sont pas toujours stable, l'équilibre peut être stable ou instable
\section{Dimension expérimentale}