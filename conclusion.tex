\chapter{Conclusion}

Nous avons tout au long de ce rapport étudié des systèmes chaotiques sous différents aspects, nous nous sommes intéressés à leurs représentations graphiques et à la façon de générer ces représentations. Nous avons tenté d'analyser mathématiquement leurs comportements afin de mieux les comprendre, cela s'est parfois avéré trop complexe à comprendre en profondeur, notament pour l'attracteur de Lorenz, mais, cela nous a permis d'explorer et de découvrir de nouveau concept et de nouvelles méthodes. Nous avons également exploré un certain nombre de pistes qui ne sont pas présentes dans ce document, comme par exemple le comportement d'un double pendule la génération d'un diagramme de bifurcation avec l'algorithme de Newton.

L'intitulé de ce document est \ogÉtude de systèmes dynamiques non déterministes\fg{}, mais dans quelle mesure est-ce exact, un aspect auxquel nous nous ne sommes pas sufisament intéressé, mais qui a son importance est la dimension statistique, par exemple dans le cas de l'attracteur de Lorenz, même si son evolution est imprévisible statistiquement parlant, il est parfaitement stable. On pourrait ainsi conclure en disant que le battement des ailes d'un papillion qui modifierais les conditions initiales d'un systèmes ne changerait pas les chances qu'une tornade ce produise au texas, mais plutot le moment où celle-ci ce produira. 