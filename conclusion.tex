\chapter{Conclusion}

Nous avons tout au long de ce rapport étudié des systèmes chaotiques
sous différent aspect, nous nous sommes intéressés à leur représentation
graphique et à la façon de générer ses représentations. Nous avons tenté d'analysé mathématiquement leur comportement
afin de mieux les comprendres, cela c'est parfois avéré trop complexe 
à comprendre en profondeur, notament pour l'attracteur de Lorenz, mais,
cela nous a permis d'explorer et de découvrir de nouveau concept et de 
nouvelles méthodes. Nous avons également éxploré un certain nombre de piste qui ne sont
pas présente dans ce document, comme par exemple le comportement d'un double pendul OU ENCORE\dots
L'intitulé de ce document est Étude de systèmes dynamiques non déterministes, mais dans qu'elle mesure est-ce
exact, un aspect auxquel nous ne somme pas sufisament intéresser mais qui a son importance
est la dimension statistique, par exemple dans le cas de l'attracteur de Lorenz, même si 
son evolution est imprévisible statistiquement paralnt il est parfaitement stable. On pourrait
ainsi conclure en disant que le battement des ailes d'un papillion qui modifierais les conditions initiales
d'un systèmes ne changerais pas les chances qu'une tornade ce produise au texas, mais plutot le moment ou celle-ci ce produira. 